\documentclass[11pt]{article}

\usepackage[utf8]{inputenc}
\usepackage[portuges]{babel}
\usepackage[T1]{fontenc}

\usepackage{mathtools}
\usepackage{amssymb}

\usepackage{multicol} % split into columns

\author{Dmitry Rocha}
\title{Limites e Derivadas - Cálculo I}

\newcommand{\sen}{{\rm sen}}
\newcommand{\tg}{{\rm tg}}
\newcommand{\cosec}{{\rm cosec}}
\newcommand{\cotg}{{\rm cotg}}

\newcommand{\limit}[3]{{
  \underset{#1 \rightarrow #2}{\lim} #3
}}

% From: http://www.wikihow.com/Write-a-Resume-in-LaTeX .
\topmargin=0in %length of margin at the top of the page (1 inch added by default)
\oddsidemargin=0in %length of margin on sides for odd pages
\evensidemargin=0in %length of margin on sides for even pages
\textwidth=6.5in %How wide you want your text to be
\marginparwidth=0.5in
\headheight=0pt %1in margins at top and bottom (1 inch is added to this value by default)
\headsep=0pt %Increase to increase white space in between headers and the top of the page
\textheight=9in %How tall the text body is allowed to be on each page

\begin{document}

\section{Limite}

\subsection{Função contínua}

\begin{description}
  \item[f contínua em p] $\leftrightarrow$
    $\underset{x\rightarrow p}{\lim } f(x) = f(p)$

  \item[f contínua em p]

    $
    \leftrightarrow
      \begin{cases}
        \forall \varepsilon > 0, \exists \delta > 0 (\delta\text{ dependendo de }
        \varepsilon\text{), tal que para todo }x \in D_f, \\
        |x-p| < \delta \to |f(x) - f(p)| < \varepsilon
      \end{cases}
    $

\end{description}

\subsection{Limite}

$$
\limit{x}{p}{f(x)} = L \leftrightarrow
  \begin{cases}
    \forall \varepsilon > 0 , \exists \delta > 0\text{ tal que, }\forall x \in D_f, \\
    0< |x-p| < \delta \to |f(x) - L| < \varepsilon
  \end{cases}
$$

\subsection{Limites laterais}

\begin{itemize}
\item
$
\limit{x}{p+}{f(x)} = L \leftrightarrow
  \begin{cases}
    \forall \varepsilon > 0 , \exists \delta > 0\text{ tal que, } \\
    p < x < p+\delta \to |f(x) - L| < \varepsilon
  \end{cases}
$

\item
$
\limit{x}{p-}{f(x)} = L \leftrightarrow
  \begin{cases}
    \forall \varepsilon > 0 , \exists \delta > 0\text{ tal que, } \\
    p - \delta < x < p \to |f(x) - L| < \varepsilon
  \end{cases}
$
\end{itemize}

\subsection{T. C.}

Seja $f(x) \leq g(x) \leq h(x)$, para $x < |x-p|<r$:

Se $\limit{x}{p}{(f)} = L = \limit{x}{p}{h(x)}$ então $\limit{x}{p}{g(x)} = L$.

\subsection{Limite fundamental}

$$\limit{x}{0}{\frac{\sen{(x)}}{x}} = 1$$

\subsection{Infinito}

\begin{description}
  \item[$+\infty$]
  $
  \limit{x}{+\infty}{f(x)} = L \leftrightarrow
    \begin{cases}
      \forall \varepsilon > 0 , \exists \delta > 0\text{ com } \delta >a\text{,
tal que} \\
      x > \delta \to L - \varepsilon < f(x)<L+\varepsilon
    \end{cases}
  $

  \item[$-\infty$]
  $
  \limit{x}{-\infty}{f(x)} = L \leftrightarrow
    \begin{cases}
      \forall \varepsilon > 0 , \exists \delta > 0\text{ com } -\delta<a\text{,
tal que} \\
      x < -\delta \to L - \varepsilon < f(x)<L+\varepsilon
    \end{cases}
  $

  \item[(a)]
  $
  \limit{x}{+\infty}{f(x)} = +\infty \leftrightarrow
    \begin{cases}
      \forall \varepsilon > 0 , \exists \delta > 0\text{ com } \delta>a\text{,
tal que} \\
      x>\delta \to f(x) > \varepsilon
    \end{cases}
  $

  \item[(b)]
  $
  \limit{x}{+\infty}{f(x)} = -\infty \leftrightarrow
    \begin{cases}
      \forall \varepsilon > 0 , \exists \delta > 0\text{ com } \delta>a\text{,
tal que} \\
      p<x<p+\delta \to f(x) > \varepsilon
    \end{cases}
  $
\end{description}

\subsection{Limite}

$$
\limit{x}{\infty}{(1+\frac{1}{x})^x} = e
$$

\section{Derivadas}

$$
f'(x) = \limit{x}{p}{ \frac{f(x) - f(p)}{x-p} =
  \limit{h}{0}{ \frac{f(p+h)- f(p)}{h} } } =
  [\frac{df}{dx}]_{x = x_0}
$$

A reta da equação $$y-f(p)=f'(p)(x-p)$$ é por definição a reta tangente ao
gráfico de $f$ no ponto $(p, f(p))$. Assim, a derivada de $f$ em $p$, é o
coeficiente angular da reta tangente ao gráfico de $f$ no ponto de abscissa
$p$.

\subsection{Derivadas}

\begin{description}
  \begin{multicols}{2}
    \item[ $f(x) = nx$ ]       $ \to f'(x) = n $
    \item[ $f(x) = x^n$ ]      $ \to f'(x) = n\times x^{(x-1)}$
    \item[ $f(x) = k$ ]        $ \to f'(x) = 0$
    \item[ $f(x) = x^{-n}$ ]   $ \to f'(x) = -n\times x^{(-n-1)} $,
      \footnote{$x\neq 0$}
    \item[ $f(x) = x^{ \frac{1}{n} }$ ] $ \to f'(x) = \frac{1}{2} \times
      x^{\frac{1}{2}-1}$,
      \footnote{$x>0$ se $n$ par e $x\neq 0$ se $n$ ímpar ($n\geq2$)}
    \item[ $f(x) = e^x$ ]      $ \to f'(x) = e^x $
    \item[ $f(x) = ln(x)$ ]    $ \to f'(x) = \frac{1}{x}$,
      \footnote{$x>0$}
    \end{multicols}
\end{description}

\subsection{Derivadas de funções trigonométricas}

\begin{itemize}
  \begin{multicols}{2}
    \item $ \sen{'(x)} = \cos{'(x)} $
    \item $ \cos{'(x)} = -\sen{'(x)} $
    \item $ \tg{'(x)} = \sec{^2(x)} $
    \item $ \sec{'(x)} = \sec{(x)}\tg{(x)}$
    \item $ \cotg{'(x)} = -\cosec{^2(x)}$
    \item $ \cosec{'(x)} = -\cosec{(x)}\times \cotg{(x)}$
  \end{multicols}
\end{itemize}

\subsection{Regras de derivação}

\begin{description}
  \item[(D1)] $(f+g)'(p) = f'(p) + g'(p)$
  \item[(D2)] $(kf)'(p) = k\times f'(p)$
  \item[(D3)] $(f\times g)'(p) = f'(p)\times g(p) + f(p)\times g'(p)$
  \item[(D4)] $( \frac{f(x)}{g(x)} )' = \frac{f'(x)\times g(x) - f(x)\times
    g'(x)}{[g(x)]^2}$
\end{description}

% \subsection{Notação de Leibniz}

% $$
% f'(x) = \frac{dy}{dx} = \limit{\Delta x}{0}{ \frac{f(x + \Delta x) -
% f(x)}{\Delta x} }
% $$

% Fazendo $\Delta y = f(x+ \Delta x)-f(x)$, resulta:

% $$
% \frac{dy}{dx} = \limit{\Delta x}{0}{ \frac{\Delta y}{\Delta x} }
% $$

\end{document}
