\documentclass[11pt]{article}

\usepackage[utf8]{inputenc}
\usepackage[portuges]{babel}
\usepackage[T1]{fontenc}

\usepackage{mathtools}
\usepackage{amssymb}

\author{Dmitry Rocha}
\title{Limites e Derivadas - Cálculo I}

\newcommand{\sen}{{\rm sen}}

\newcommand{\limit}[3]{{
  \underset{#1 \rightarrow #2}{\lim} #3
}}

% From: http://www.wikihow.com/Write-a-Resume-in-LaTeX .
\topmargin=0in %length of margin at the top of the page (1 inch added by default)
\oddsidemargin=0in %length of margin on sides for odd pages
\evensidemargin=0in %length of margin on sides for even pages
\textwidth=6.5in %How wide you want your text to be
\marginparwidth=0.5in
\headheight=0pt %1in margins at top and bottom (1 inch is added to this value by default)
\headsep=0pt %Increase to increase white space in between headers and the top of the page
\textheight=9in %How tall the text body is allowed to be on each page

\begin{document}

\section{Função contínua}

\begin{description}
  \item[f contínua em p] $\leftrightarrow$
  $\underset{x\rightarrow p}{\lim } f(x) = f(p)$

  \item[f contínua em p]

$
\leftrightarrow
  \begin{cases}
    \forall \varepsilon > 0, \exists \delta > 0 (\delta\text{ dependendo de }
    \varepsilon\text{), tal que para todo }x \in D_f, \\
    |x-p| < \delta \to |f(x) - f(p)| < \varepsilon
  \end{cases}
$

\end{description}

\section{Limite}

$
\limit{x}{p}{f(x)} = L \leftrightarrow
  \begin{cases}
    \forall \varepsilon > 0 , \exists \delta > 0\text{ tal que, }\forall x \in D_f, \\
    0< |x-p| < \delta \to |f(x) - L| < \varepsilon
  \end{cases}
$

\section{Limites laterais}

\begin{itemize}
\item
$
\limit{x}{p+}{f(x)} = L \leftrightarrow
  \begin{cases}
    \forall \varepsilon > 0 , \exists \delta > 0\text{ tal que, } \\
    p < x < p+\delta \to |f(x) - L| < \varepsilon
  \end{cases}
$

\item
$
\limit{x}{p-}{f(x)} = L \leftrightarrow
  \begin{cases}
    \forall \varepsilon > 0 , \exists \delta > 0\text{ tal que, } \\
    p - \delta < x < p \to |f(x) - L| < \varepsilon
  \end{cases}
$
\end{itemize}

\section{T. C.}

Seja $f(x) \leq g(x) \leq h(x)$, para $x < |x-p|<r$:

Se $\limit{x}{p}{(f)} = L = \limit{x}{p}{h(x)}$ então $\limit{x}{p}{g(x)} = l$.

\section{Limite fundamental}

$\limit{x}{0}{\frac{\sen{(x)}}{x}} = 1$

\section{Infinito}

\begin{description}
  \item[$+\infty$]
  $
  \limit{x}{+\infty}{f(x)} = L \leftrightarrow
    \begin{cases}
      \forall \varepsilon > 0 , \exists \delta > 0\text{ com } \delta >a\text{,
tal que} \\
      x > \delta \to L - \varepsilon < f(x)<L+\varepsilon
    \end{cases}
  $

  \item[$-\infty$]
  $
  \limit{x}{-\infty}{f(x)} = L \leftrightarrow
    \begin{cases}
      \forall \varepsilon > 0 , \exists \delta > 0\text{ com } -\delta<a\text{,
tal que} \\
      x < -\delta \to L - \varepsilon < f(x)<L+\varepsilon
    \end{cases}
  $

  \item[(a)]
  $
  \limit{x}{+\infty}{f(x)} = +\infty \leftrightarrow
    \begin{cases}
      \forall \varepsilon > 0 , \exists \delta > 0\text{ com } \delta>a\text{,
tal que} \\
      x>\delta \to f(x) > \varepsilon
    \end{cases}
  $

  \item[(b)]
  $
  \limit{x}{+\infty}{f(x)} = -\infty \leftrightarrow
    \begin{cases}
      \forall \varepsilon > 0 , \exists \delta > 0\text{ com } \delta>a\text{,
tal que} \\
      p<x<p+\delta \to f(x) > \varepsilon
    \end{cases}
  $
\end{description}

\section{Limite}

$
\limit{x}{\infty}{(1+\frac{1}{x})^x} = e
$

\end{document}
